\documentclass[12pt]{article}
\usepackage[utf8]{inputenc}
\usepackage[russian]{babel}
\usepackage[margin=1.5in,left=1cm,right=1cm, top=2cm,bottom=2cm,bindingoffset=0cm]{geometry}
\usepackage{graphicx}
\usepackage{color}
\usepackage{amssymb}
\usepackage{minted}
\usepackage{hyperref}
\usepackage{amsmath}
\usepackage{fancyhdr}



\title{Title}
\author{Амеличев Константин, ПМИ 191.}
\date{Date}

\newcommand{\problem}[2]{

\section {Задача #1}
\textbf {Постановка задачи.} {#2}

\textbf {Решение.}
}

\newcommand{\limit}[2]{\displaystyle \lim_{#1 \to #2}}

\newcommand{\rangesum}[2]{\displaystyle \sum_{#1}^{#2}}

\newcommand{\mintedparams}{
% frame=lines
% framesep=2mm,
% baselinestretch=1.2,
% bgcolor=LightGray    
}

\pagestyle{fancy}
\fancyhf{}
\fancyhead[LE,RO]{Амеличев Константин, ПМИ 191, @kik0s, \href{http://github.com/kik0s}{\textcolor{blue}{github}}, \href{http://codeforces.com/profile/kikos}{\textcolor{blue}{codeforces}}, \href{http://vk.com/i_tried_to_name_myself_kikos}{\textcolor{blue}{vk}}}
\fancyhead[RE,LO]{Лекция АиСД 03.12}

\begin{document}

\paragraph{Хэширование}

Есть задача сравнения объектов:

$$U = \{objects\},\ u, v \in U:\ u \neq v?$$

Введем функцию $h:\ U \rightarrow \mathbb{Z}_m$, такую что $\forall\ u, v \in U:\ u \sim v \rightarrow h(v) = h(u)$. Обычно $m \ll |U|$.

Зачем юзать хэши, а не наивное сравнение?
\begin{itemize}
\item Бывает много сравнений, и мы не хотим дублировать вычисления
\item Безопасность
\item Для некоторых хешей верно, что $h(f(v, u)) = g(h(v), h(u))$. То есть можно вычислить хэш от некоего объекта, пользуясь уже посчитанными хэшами для других объектов.
\item Иногда мы сравниваем объекты не на равенство, а на изоморфность.
\item Сравнивать объекты бывает дорого
\end{itemize}

Требования к хэшу:
\begin{itemize}
    \item вычислим за линейное время
    \item детерменирован
    \item семейство хэш-функций $\mathbb{H}$ (и можем взять сколько угодно оттуда)
    \item равномерность $\forall_{v \neq u} v,\ u:\ p_{h \in \mathbb{H}}(h(u) = h(v)) \simeq \frac{1}{m}$
    \item масштабируемость
    \item необратимость
    \item (\textit{optional}) лавинный эффект (при маленьком изменении объекта хэш меняется сильно) 
\end{itemize}

Важно, что мы хотим брать случайную функцию из семейства $\mathbb{H}$ на момент старта программы, потому что псевдослучайная функция на самом деле нам дает детерменированный алгоритм, который неверен.

\paragraph{Идеальная хэш-функция}

Так как объектов счетно, то отсортируем их, далее для каждого объекта запомним случайную величину от 1 до $m$ (или даже можем запоминать ее лениво!).

\paragraph{Полиномиальный хэш}

Сводим объекты к строкам и хэшируем строки:

$$s = c_0c_2c_3 \dots c_{n-1},\ c_i > 0$$

$$h_b(s) = \rangesum{i=0}{n-1} c_i \cdot b^i\ (mod\ m)$$

$$p_b(h_b(s_1) = h_b(s_2)) \le \frac{n}{m}\ \text{для фиксированного m}$$

Доказательство: Рассмотрим функцию как многочлен, теперь для равных функций смотрим на то, равна ли разность многочленов нулю для какого-то $b$. У многочлена от $b$ степень равна $n$, а вероятность попасть в конкретный модуль $\frac{1}{m}$.

$$h(s_1 + s_2) = h(s_1) + h(s_2) \cdot b^{|s_1|}$$

\end{document}
