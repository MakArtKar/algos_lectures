\documentclass[12pt]{article}
\usepackage[utf8]{inputenc}
\usepackage[russian]{babel}
\usepackage[margin=1.5in,left=1cm,right=1cm, top=2cm,bottom=2cm,bindingoffset=0cm]{geometry}
\usepackage{graphicx}
\usepackage{color}
\usepackage{amssymb}
\usepackage{minted}
\usepackage{hyperref}
\usepackage{amsmath}
\usepackage{fancyhdr}



\title{Title}
\author{Амеличев Константин, ПМИ 191.}
\date{Date}

\newcommand{\problem}[2]{

\section {Задача #1}
\textbf {Постановка задачи.} {#2}

\textbf {Решение.}
}

\newcommand{\limit}[2]{\displaystyle \lim_{#1 \to #2}}

\newcommand{\rangesum}[2]{\displaystyle \sum_{#1}^{#2}}

\newcommand{\rangeint}[2]{\displaystyle \int_{#1}^{#2}}


\newcommand{\mintedparams}{
% frame=lines
% framesep=2mm,
% baselinestretch=1.2,
% bgcolor=LightGray    
}

\pagestyle{fancy}
\fancyhf{}
\fancyhead[LE,RO]{Амеличев Константин, ПМИ 191, @kik0s, \href{http://github.com/kik0s}{\textcolor{blue}{github}}, \href{http://codeforces.com/profile/kikos}{\textcolor{blue}{codeforces}}, \href{http://vk.com/i_tried_to_name_myself_kikos}{\textcolor{blue}{vk}}}
\fancyhead[RE,LO]{Лекция АиСД 20.02}

\begin{document}

\paragraph{2-sat.} Автор опоздал на эту часть лекции, поэтому напишет ее сам позднее.

\paragraph{Мосты и точки сочленения.} Назовем мостами такие ребра, при удалении которых граф теряет связность. Аналогичные вершины определим как точки сочленения. Как их найти? Сделаем dfs в неориентированном графе, и построем дерево dfs. Те ребра, по которым мы не переходили, мы назовем обратными. В дереве dfs эти ребра будут идти из вершины в ее какого-то предка.

Что тогда верно про мосты? Из поддерева ребра-моста нет ни одного обратного ребра, которое шло бы выше, чем наше ребро. А найти самое высокое ребро в поддереве можно обычной динамикой. Аналогично с точками сочленения. 

\paragraph{Отношения вершинной и реберной двусвязности.} Проще всего запомнить так --- отношения вершинной двусвязности это отношение на ребрах, а отношение реберной --- отношение на вершинах.

Вершины $v$ и $u$ находятся в одной компоненте реберной двусвязности, если существует реберно-простой цикл, содержащий $u$ и $v$.

Отношение такой двусвязности транзитивно. Можно показать, что если $u \equiv v$ и $v \equiv w$ (при этом $v$ и $w$ соединены ребром), то и $u \equiv w$. Для этого надо склеить два цикла, и найти в них новый --- из $u$ в $w$.

Чтобы найти классы эквивалентности, можно удалить все мосты в графе.

Отношение вершинной двусвязности определяется аналогично, но на ребрах.

\paragraph{Задача о поиске кратчайшего пути.}

Пусть нам дан граф $G = (V,\ E)$. Каждому ребру задан какой-то вес, а весу пути соответствует суммарный вес всех ребер на пути.

Веса на ребрах могут быть естественными (неотрицательные), или не естественными (разрешаем отрицательный вес ребер). Также отдельный случай для нас --- существование циклов отрицательного веса. Кроме того, мы иногда хотим искать любые пути, иногда реберно простые, иногда вершинно простые.

\paragraph{Bfs.} Пусть мы хотим найти кратчайший путь, где вес каждого ребра равен 1. Идея такая --- мы хотим построить слоистую декомпозицию, где уровню $i$ соответствуют вершины на расстоянии $i$ от стартовой. Алгоритм реализуется на очереди.



\end{document}
