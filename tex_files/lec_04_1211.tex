\documentclass[12pt]{article}
\usepackage[utf8]{inputenc}
\usepackage[russian]{babel}
\usepackage[margin=1.5in,left=1cm,right=1cm, top=2cm,bottom=2cm,bindingoffset=0cm]{geometry}
\usepackage{graphicx}
\usepackage{color}
\usepackage{amssymb}
% \usepackage{minted}
\usepackage{hyperref}
\usepackage{fancyhdr}



\title{Title}
\author{Амеличев Константин, ПМИ 191.}
\date{Date}

\newcommand{\problem}[2]{

\section {Задача #1}
\textbf {Постановка задачи.} {#2}

\textbf {Решение.}
}

\newcommand{\limit}[2]{\displaystyle \lim_{#1 \to #2}}

\newcommand{\rangesum}[2]{\displaystyle \sum_{#1}^{#2}}

\newcommand{\mintedparams}{
% frame=lines
% framesep=2mm,
% baselinestretch=1.2,
% bgcolor=LightGray    
}

\pagestyle{fancy}
\fancyhf{}
\fancyhead[LE,RO]{Амеличев Константин, ПМИ 191, @kik0s, \href{http://github.com/kik0s}{\textcolor{blue}{github}}, \href{http://codeforces.com/profile/kikos}{\textcolor{blue}{codeforces}}, \href{http://vk.com/i_tried_to_name_myself_kikos}{\textcolor{blue}{vk}}}
\fancyhead[RE,LO]{Лекция по АиСД 12.11}

\begin{document}

Способы оценки времени работы:
\begin{itemize}
\item Прямой учет

\item Рекурсивная оценка

\item Амортизационный анализ
\end{itemize}

\paragraph{Прямой учет}

Время работы строчки --- произведение верхних оценок по всем строчкам-предкам нашей.

К примеру

\begin{verbatim}
while (!is_sorted()) { // O(# inversions) = O(n^2)
    for (int i = 0; i + 1 < n; i++) { // O(n) * O(parent) = O(n^3)
        if (a[i] > a[i + 1]) {
            swap(a[i], a[i + 1]); // O(n^3)
        }
    }
}
\end{verbatim	}

\paragraph{Рекурсивная оценка}

Пример --- сортировка слиянием

Нас интересует две вещи: инвариант и переход. Для оценки времени используем рекурренту вида $T(n) = O(f(n)) + \rangesum{n' \in calls}{} T(n')$ 

При этом если мы доказываем время работы, то показываем $T(n) \le c \cdot f(n)$, зная, что для $n'\ \exists c:\ T(n') \le c \cdot f(n)$

Важно, что $c$  глобальное и не должно увеличиваться в ходе доказательства


\paragraph{stable sort}
Делает сортировку, не меняя порядок равных элементов относительно исходной последовательности. Merge-sort стабилен.
\paragraph{inplace-algorithm}
Не требует дополнительной памяти и делает все прямо на данной памяти (у нас есть $\log n$ памяти на рекурсию). Quick-sort inplace.

\paragraph{Время работы qsort}

$$T(n) = \max_{input} average_{rand} t(input, rand) = \max_{|input|=n} E t(input)$$

$$T(n) \le \Theta(n) + \frac{1}{n} \rangesum{k=0}{n-1} (T(k + 1) + T(n - k)) \le \Theta(n) + \frac{2}{n} \cdot \rangesum{k=1}{n} T(k - 1) \le a \cdot n + \frac{2}{n} \rangesum{k=1}{n} c \cdot (k - 1) \cdot \log (k - 1) \le$$

$$\le a \cdot n + \frac{2}{n} \rangesum{k=1}{\frac{n}{2}} c \cdot (k - 1) \cdot \log n - \frac{2}{n} \rangesum{k=1}{\frac{n}{2}} c \cdot (k - 1) + \frac{2}{n} \rangesum{k=\frac{n}{2} + 1}{n} c \cdot (k - 1) \cdot \log n \le$$

$$\le a \cdot n + \frac{2}{n} \cdot n^2 \cdot \log n - \frac{2c}{n} \cdot \frac{(n - 2)^2}{4} \le a \cdot n + cn \log n - \frac{c (n - 2)}{4} \le cn \log n$$

$$\frac{c (n - 2)}{4} \ge a \cdot n$$

$$c \cdot n - 2c \ge 4 \cdot a \cdot n$$

$$c \ge \frac{4 \cdot a \cdot n}{(n - 2)}$$, что верно для достаточно больших $n$.

\paragraph{Ограничение на число сравнений в сортировке}

Бинарные сравнения на меньше.

Рассмотрим дерево переходов. Для перестановки есть хотя бы один лист --- листьев котя бы $n!$

$$L(T) \le 2^x, d(T) \le x$$, если $x$ --- ответ


$$L(T) \ge n!$$

$$d(T) \ge \log n! \ge \log{\frac{n}{2}}^{\frac{n}{2}} = \log 2^{(\log \frac{n}{2}) \cdot \frac{n}{2}} = \frac{n}{2} \cdot \log \frac{n}{2} = \Omega (n \log n)$$

\end{document}