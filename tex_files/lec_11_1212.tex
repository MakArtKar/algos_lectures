\documentclass[12pt]{article}
\usepackage[utf8]{inputenc}
\usepackage[russian]{babel}
\usepackage[margin=1.5in,left=1cm,right=1cm, top=2cm,bottom=2cm,bindingoffset=0cm]{geometry}
\usepackage{graphicx}
\usepackage{color}
\usepackage{amssymb}
\usepackage{minted}
\usepackage{hyperref}
\usepackage{amsmath}
\usepackage{fancyhdr}



\title{Title}
\author{Амеличев Константин, ПМИ 191.}
\date{Date}

\newcommand{\problem}[2]{

\section {Задача #1}
\textbf {Постановка задачи.} {#2}

\textbf {Решение.}
}

\newcommand{\limit}[2]{\displaystyle \lim_{#1 \to #2}}

\newcommand{\rangesum}[2]{\displaystyle \sum_{#1}^{#2}}

\newcommand{\mintedparams}{
% frame=lines
% framesep=2mm,
% baselinestretch=1.2,
% bgcolor=LightGray    
}

\pagestyle{fancy}
\fancyhf{}
\fancyhead[LE,RO]{Амеличев Константин, ПМИ 191, @kik0s, \href{http://github.com/kik0s}{\textcolor{blue}{github}}, \href{http://codeforces.com/profile/kikos}{\textcolor{blue}{codeforces}}, \href{http://vk.com/i_tried_to_name_myself_kikos}{\textcolor{blue}{vk}}}
\fancyhead[RE,LO]{Лекция АиСД 12.12}

\begin{document}

\paragraph{Деревья поиска}
\hspace{\fill}
Храним структуру данных, которая должна уметь делать все то же самое, что можно делать с отсортированным массивом, но еще с добавлением-удалением.

\paragraph{Binaary search tree (BST)}

Корневое дерево. В вершине храним левого сына, правого сына, предка, ключ.

Обозначения:
\begin{itemize}

\item $l(v)$ --- левый сын

\item $r(v)$ --- правый сын

\item $p(v)$ --- предок

\item $key(v)$ --- ключ

\item $T(v)$ --- поддерево

\item $S(v) = |T(v)|$ --- размер

\item $A(v)$ --- множество предков

\item $seg(v) = [\min_{u \in T(v)} key(u); \max_{u \in T(v)} key(u)]$

\end{itemize}

Условие BST: 

$$ \forall u \in T(l(v)):\ key(u) < key(v)$$

$$ \forall u \in T(r(v)):\ key(u) > key(v)$$

То есть, ключи лежат в порядке лево-правого обхода (в порядке <<выписать левое поддерево - выписать вершину - выписать правое поддерево>>)

$v \in seg(u) \leftrightarrow v \in T(u)$

Поиск в дереве --- надо сделать спуск. То есть, если текущая вершина --- не та, которая нам нужна, то можно понять, где лежит нужная нам, с помощью условия BST.

Нахождение следующего --- либо спуск вправо, либо подъем по предкам до первого большего.

Основная проблема --- операции вставки/удаления, которые должны делать дерево сбалансированным (таким, высота которого нас устраивает, то есть примерно $\log n$)

\paragraph{Декартово дерево}

У каждой вершины будем хранить не только ключ, но и какой-то приоритет $y$. Построим дерево так, что по $y$ это куча, а по $x$ это BST.

Тогда декартово дерево задается однозначно, если определить все приоритеты. Почему? Расставим точки на плоскости в соответствии с $(x, y)$, после чего найдем корень. У корня будет наименьший приоритет и поэтому он определяется единственным образом (будем считать, что приоритеты различны). Тогда к корню нужно приписать слева и справа по дереву, которые рекурсивно строятся в левой и правых частях.

\textbf{Утверждение} Если взять случайные $y$, то матожидание глубины дерева $O(\log n)$

\textbf{Доказательство:} Нет.

\textbf{Утверждение} Если взять случайные $y$, то матожидание глубины каждой вершины $O(\log n)$

\textbf{Доказательство.} Матожидание высоты вершины --- это число вершин, которые являются ее предками. Вершина $i$ будет предком вершины $j$, если $y_i = \max \{y_i, y_{i+1}, \dots, y_j\}$. Матожидание суммы таких величин для $i$ это $\rangesum{j=0}{i-1} \frac{1}{i-j} + \rangesum{j=i+1}{n-1} \frac{1}{j-i} \le 2 \rangesum{i=1}{n} \frac{1}{i} = O(\log n)$


\end{document}
