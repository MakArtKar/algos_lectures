\documentclass[12pt]{article}
\usepackage[utf8]{inputenc}
\usepackage[russian]{babel}
\usepackage[margin=1.5in,left=1cm,right=1cm, top=2cm,bottom=2cm,bindingoffset=0cm]{geometry}
\usepackage{graphicx}
\usepackage{color}
\usepackage{amssymb}
\usepackage{minted}
\usepackage{hyperref}
\usepackage{amsmath}
\usepackage{fancyhdr}



\title{Title}
\author{Амеличев Константин, ПМИ 191.}
\date{Date}

\newcommand{\problem}[2]{

\section {Задача #1}
\textbf {Постановка задачи.} {#2}

\textbf {Решение.}
}

\newcommand{\limit}[2]{\displaystyle \lim_{#1 \to #2}}

\newcommand{\rangesum}[2]{\displaystyle \sum_{#1}^{#2}}

\newcommand{\mintedparams}{
% frame=lines
% framesep=2mm,
% baselinestretch=1.2,
% bgcolor=LightGray    
}
\pagestyle{fancy}
\fancyhf{}
\fancyhead[LE,RO]{Амеличев Константин, ПМИ 191, @kik0s, \href{http://github.com/kik0s}{\textcolor{blue}{github}}, \href{http://codeforces.com/profile/kikos}{\textcolor{blue}{codeforces}}, \href{http://vk.com/i_tried_to_name_myself_kikos}{\textcolor{blue}{vk}}}
\fancyhead[RE,LO]{Алгоритмы 14.11}

\begin{document}

\section{Сортировки основанные на внутреннем виде данных}

Имеем $n$ чисел $[0, U - 1],\ U = 2^w$, числа укладываются в RAM-модель

\paragraph{Сортировка подсчетом}

Заводим массив $cnt[U],\ cnt[x] = |\{i : a_i = x\}|$. Дальше переводим $count \rightarrow pref,\ pref[x] = pref[x - 1] + count[x]$

$O(n + U)$

\paragraph{Поразрядная сортировка}

$b_{ij}$---$j$-й бит $i$-го числа. (Сортируем бинарные строки длины $w$)

Поочередно сортируем строки, разбивая их на классы эквивалентности по $2^{iter}$ последним символам. После чего мы стабильно сортируем по $(iter+1)$-му символу. 

$O(\log_n U n)$

\paragraph{Bucket sort}

Разбиваем множество на корзины, каждой корзине соответствует отрезок. В каждой корзине запускаемся рекурсивно. 

$O(n \log U)$

В продакшне используют первую пару итераций, чтобы сильно снизить размерность на реальных данных.

Пусть мы хотим отсортить равновероятные числа из  $[0, 1]$. В каждом бакете отсортируем за квадрат. Получим $O(n)$. 

$$t(n) \le \rangesum{i=1}{n} c \cdot (1 + E (cnt_i)^2) \le c \cdot n + c \cdot \rangesum{i=1}{n} E (cnt_i^2) = $$
$$ = c \cdot n + c \cdot \rangesum{i=1}{n} \rangesum{j=1}{n} E I_{A_{ij}} = c \cdot n + c \cdot n^2 \cdot \frac{1}{n} \le 2 \cdot c \cdot n$$

\section{Иерархия памяти}

Нас интересует задержка (latency), пропускная способность (throughput). Подгрузка $x$ данных занимает $l + \frac{x}{t}$

От долгой к быстрой
\begin{enumerate}
\item external machine / internet
\item HDD
\item SSD
\item RAM
\item L3
\item L2
\item L1
\item registers
\end{enumerate}

\section{Алгоритмы во внешней памяти}

$n$ --- размер задачи

$M$ --- размер $RAM$

$B$ --- блок данных

$B \ll M \ll n$

$\log n \ll B$

$B < \sqrt{M}$ (но это неявно и не факт)

\paragraph{Mergesort во внешней памяти}

Обычный mergesort, но три типа событий в $merge$:
\begin{enumerate}
    \item Кончился первый буфер --- подгружаем новый
    \item Кончился второй --- аналогично
    \item Кончился буфер для слияния --- выписываем обратно в RAM и сбрасываем
\end{enumerate}

$O(\frac{n}{B} \log n) \rightarrow O(\frac{n}{B} \log {\frac{n}{B}}) \rightarrow O(\frac{n}{B} \log_{\frac{M}{B}} {\frac{n}{B}})$

+2 идеи: \begin{enumerate}
\item Дошли до размера $M$ --- явно посортим в RAM
\item Можем сливать сразу $\frac{M}{B}$ массивов
\end{enumerate}

\end{document}
