\documentclass[12pt]{article}
\usepackage[utf8]{inputenc}
\usepackage[russian]{babel}
\usepackage[margin=1.5in,left=1cm,right=1cm, top=2cm,bottom=2cm,bindingoffset=0cm]{geometry}
\usepackage{graphicx}
\usepackage{color}
\usepackage{amssymb}
\usepackage{minted}
\usepackage{hyperref}
\usepackage{amsmath}
\usepackage{fancyhdr}



\title{Title}
\author{Амеличев Константин, ПМИ 191.}
\date{Date}

\newcommand{\problem}[2]{

\section*{Задача #1}
\textbf {Постановка задачи.} {#2}

\textbf {Решение.}
}

\newcommand{\limit}[2]{\displaystyle \lim_{#1 \to #2}}

\newcommand{\rangesum}[2]{\displaystyle \sum_{#1}^{#2}}

\newcommand{\rangeint}[2]{\displaystyle \int_{#1}^{#2}}


\newcommand{\mintedparams}{
% frame=lines
% framesep=2mm,
% baselinestretch=1.2,
% bgcolor=LightGray    
}

\pagestyle{fancy}
\fancyhf{}
\fancyhead[LE,RO]{Амеличев Константин, ПМИ 191, @kik0s, \href{http://github.com/kik0s}{\textcolor{blue}{github}}, \href{http://codeforces.com/profile/kikos}{\textcolor{blue}{codeforces}}, \href{http://vk.com/i_tried_to_name_myself_kikos}{\textcolor{blue}{vk}}}
\fancyhead[RE,LO]{Лекция АиСД 30.04}

\begin{document}

\paragraph{Ахо-Корасик.} Хотим решать следующую задачу: Дан набор из $n$ строк $s_i$ суммарного размера $L$ на алфавите $\sigma$. Кроме этого, дан текст $t$, и задается вопрос насчет вхождений: сколько строк из набора входят в $t$ как подстроки?

Первый шаг: построить бор на наборе строк. Теперь мы хотим сделать на этом боре детерменированный автомат. Автомат при чтении $t_{0}\ldots t_{i}$ должен переводить нас в вершину, которой соответствует самый длинный суффикс строки $t_{0}\ldots t_{i}$ из присутствующих в боре.

Для каждой вершины $v$ создадим суффиксную ссылку $link$, которая будет вести в вершину, соответствующую самому длинному суффиксу $str(v)$, не совпадающему с $str(v)$, но присутствующему в боре. Кроме этого, обозначим переходы автомата за $go_c$. Тогда понятно, что $link_v$ можно найти, если посмотреть на суффиксную ссылку предка. А именно, если из суффиксной ссылки предка есть переход по символу, который написан на ребре $(p_v, v)$, то $link_v$ ведет именно туда. Если там перехода не было, то надо посмотреть на следующего предка в дереве суфссылок, и сделать аналогичную операцию. Но, если считать, что для предков посчитано $go_c$, то это то же самое, что взять $go_{c(p_v, v)}$ для $link_{p_v}$.

Для того, чтобы считать переходы в автомате, тоже можно воспользоваться предыдущими значениями. А именно, если из $v$ есть ребро с символом $c$, то тогда $go_c$ будет указывать в конец ребра. Иначе следует посмотреть на $go_c$ для нашей суффиксной ссылки.

Обе динамики можно посчитать либо лениво, либо с помощью обхода в порядке возрастания глубин.

\paragraph{Оценка времени работы} Можно оценить как $O(m |\sigma|)$, где $m$ --- число вершин в дереве, потому что динамика занимает ровно столько памяти (тривиально), и для нее верно $time = memory$. Также есть оценка $O(L)$, которую можно получить, если показать, что у нас всего спусков бывает не более чем $O(L)$, а подъемов суммарно не больше, чем спусков.


\end{document}
