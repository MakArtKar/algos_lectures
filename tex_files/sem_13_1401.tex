\documentclass[12pt]{article}
\usepackage[utf8]{inputenc}
\usepackage[russian]{babel}
\usepackage[margin=1.5in,left=1cm,right=1cm, top=2cm,bottom=2cm,bindingoffset=0cm]{geometry}
\usepackage{graphicx}
\usepackage{color}
\usepackage{amssymb}
\usepackage{minted}
\usepackage{hyperref}
\usepackage{amsmath}
\usepackage{fancyhdr}



\title{Title}
\author{Амеличев Константин, ПМИ 191.}
\date{Date}

\newcommand{\problem}[2]{

\section {Задача #1}
\textbf {Постановка задачи.} {#2}

\textbf {Решение.}
}

\newcommand{\limit}[2]{\displaystyle \lim_{#1 \to #2}}

\newcommand{\rangesum}[2]{\displaystyle \sum_{#1}^{#2}}

\newcommand{\mintedparams}{
% frame=lines
% framesep=2mm,
% baselinestretch=1.2,
% bgcolor=LightGray    
}

\pagestyle{fancy}
\fancyhf{}
\fancyhead[LE,RO]{Амеличев Константин, ПМИ 191, @kik0s, \href{http://github.com/kik0s}{\textcolor{blue}{github}}, \href{http://codeforces.com/profile/kikos}{\textcolor{blue}{codeforces}}, \href{http://vk.com/i_tried_to_name_myself_kikos}{\textcolor{blue}{vk}}}
\fancyhead[RE,LO]{Семинар 14.01}

\begin{document}

\paragraph{Splay tree.}

Уже пройденные способы балансировки для $BST$, которые у нас были, это способы из ДД (четкая структура и ожидаемая глубина) и 2-3 дерева (гарантированная одинаковая глубина для всех вершин).

Новый способ балансировки --- не напрягаться с балансированием лишний раз, и работать за амортизированное время.

Основной нашей операцией будет $expose(v)$, которая будет превращать вершину $v$ в корень дерева. Условие $BST$ на дереве сохраняется. Эта операция будет поддерживать два типа поворотов: левый и правый. Поворот вершины $v$ возьмет ребро от $v$ к предку $u$, и его <<повернет>> ~--- если $v$ было левым сыном $u$, то $u$ станет правым сыном $v$. При этом правый сын $v$ станет левым сыном $u$. Другой поворот делается симметрично. Вершина $v$ станет ближе к корню дерева. Если применять повороты к $v$, пока можно, то в итоге $v$ станет корнем.

$expose$ будет выполняться после каждой операции, и работать за высоту дерева. Более того, поскольку все остальное тоже работает за высоту дерева, то мы будем оценивать только $expose$.

Каждый раз делать повороты нельзя, потому что будет работать за долго. У нас будет три операции, которые будут поднимать нашу вершину:
\begin{itemize}
\item zig
\item zig-zig
\item zig-zag
\end{itemize}

\paragraph{zig.} Самая тупая операция --- если мы сын предка, то делаем поворот. Оставшиеся операции будут поднимать нас сразу на 2.

\paragraph{zig-zig.} Если наш дедушка от нас находится справа-справа или слева-слева, то выполним сначала поворот предка, а потом себя. 

\paragraph{zig-zag.} Иначе мы сначала сделаем поворот себя, а потом поворот предка.

\textit{Лучше порисовать картинки или погуглить визуализацию, потому что иначе будет очень непонятно, что сейчас произошло.}

Также можно выразить $split$ и $merge$ через предыдущие операции.

\paragraph{Потенциал.} Введем $\Phi(t) = \displaystyle \sum_{v \in t} rank(v) = \displaystyle \sum_{v \in t} \log_2 {size(v)}$. Тогда стоимость $expose$ это 

$$\tilde t_e = t_e + \Phi(t') - \Phi(t) \le 3(rank(root) - rank(v)) + 1$$

Воспользуемся $rank(root) - rank(v) = (rank(root) - rank(u_1)) + (rank(u_1) - rank(u_2)) + \dots + (rank(u_n) - rank(v))$ и разложим $expose$ на три операции, и покажем, что они тоже оплачиваются потенциалом.

\end{document}
