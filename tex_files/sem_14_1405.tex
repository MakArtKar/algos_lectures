\documentclass[12pt]{article}
\usepackage[utf8]{inputenc}
\usepackage[russian]{babel}
\usepackage[margin=1.5in,left=1cm,right=1cm, top=2cm,bottom=2cm,bindingoffset=0cm]{geometry}
\usepackage{graphicx}
\usepackage{color}
\usepackage{amssymb}
\usepackage{minted}
\usepackage{hyperref}
\usepackage{amsmath}
\usepackage{fancyhdr}



\title{Title}
\author{Амеличев Константин, ПМИ 191.}
\date{Date}

\newcommand{\problem}[2]{

\section*{Задача #1}
\textbf {Постановка задачи.} {#2}

\textbf {Решение.}
}

\newcommand{\limit}[2]{\displaystyle \lim_{#1 \to #2}}

\newcommand{\rangesum}[2]{\displaystyle \sum_{#1}^{#2}}

\newcommand{\rangeint}[2]{\displaystyle \int_{#1}^{#2}}


\newcommand{\mintedparams}{
% frame=lines
% framesep=2mm,
% baselinestretch=1.2,
% bgcolor=LightGray    
}

\pagestyle{fancy}
\fancyhf{}
\fancyhead[LE,RO]{Амеличев Константин, ПМИ 191, @kik0s, \href{http://github.com/kik0s}{\textcolor{blue}{github}}, \href{http://codeforces.com/profile/kikos}{\textcolor{blue}{codeforces}}, \href{http://vk.com/i_tried_to_name_myself_kikos}{\textcolor{blue}{vk}}}
\fancyhead[RE,LO]{Семинар АиСД 14.05}

\begin{document}

\paragraph{Формальные грамматики и языки.} Обозначим алфавит символов за $\sigma$, а множество конечных строк за $L \subset 2^{\sigma^*}$, где $\sigma^*$ --- это все последовательности конечной длины.

\paragraph{Регулярные выражения.} Тривиальные регулярные выражения: $\empty$ (пустое множество) $\epsilon$ (пустая строка), $x \in \sigma$. Остальные регулярные выражения определяются рекурсивно относительно них. А именно, мы также разрешаем $A + B$ (последовательная запись), $A | B$ (выбор из двух регулярок), $A*$ (повторение строки из языка $A$ $0, 1, 2, \ldots$ раз) Звездочка еще называется замыканием $Kleene$.

Например, $\{0|1|2|3\}*+\{0|1|2|3\}$ это множество всех непустых строк из символов $\sigma=\{0, 1, 2, 3\}$.

\paragraph{Способы задания языков.} Регулярные выражения, ДКА, НКА,  $\epsilon$-НКА. Эти способы эквивалентны. Очевидно $L_{\text{ДКА}} \subset L_{\text{НКА}} \subset L_{\text{$\epsilon$-НКА}}$ Вложенность $\epsilon$-НКА в  ДКА можно показать, если рассмотреть алгоритм приведения одного автомата в другой. Например, можно выделить все подмножества вершин НКА как отдельные вершины ДКА (то есть, он будет иметь экспоненциальный размер). Тогда переход по ребру --- это то же самое, что взять все вершины подмножества, и объединить переходы по ребрам из них. Еще стоит следить за переходами по пустому символу, потому что вершина сразу задает подмножество вершин, достижимых из нее по $\epsilon$-ребрам.



\end{document}
