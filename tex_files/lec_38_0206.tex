\documentclass[12pt]{article}
\usepackage[utf8]{inputenc}
\usepackage[russian]{babel}
\usepackage[margin=1.5in,left=1cm,right=1cm, top=2cm,bottom=2cm,bindingoffset=0cm]{geometry}
\usepackage{graphicx}
\usepackage{color}
\usepackage{amssymb}
\usepackage{minted}
\usepackage{hyperref}
\usepackage{amsmath}
\usepackage{fancyhdr}



\title{Title}
\author{Амеличев Константин, ПМИ 191.}
\date{Date}

\newcommand{\problem}[2]{

\section*{Задача #1}
\textbf {Постановка задачи.} {#2}

\textbf {Решение.}
}

\newcommand{\limit}[2]{\displaystyle \lim_{#1 \to #2}}

\newcommand{\rangesum}[2]{\displaystyle \sum_{#1}^{#2}}

\newcommand{\rangeint}[2]{\displaystyle \int_{#1}^{#2}}


\newcommand{\mintedparams}{
% frame=lines
% framesep=2mm,
% baselinestretch=1.2,
% bgcolor=LightGray    
}

\pagestyle{fancy}
\fancyhf{}
\fancyhead[LE,RO]{Амеличев Константин, ПМИ 191, @kik0s, \href{http://github.com/kik0s}{\textcolor{blue}{github}}, \href{http://codeforces.com/profile/kikos}{\textcolor{blue}{codeforces}}, \href{http://vk.com/i_tried_to_name_myself_kikos}{\textcolor{blue}{vk}}}
\fancyhead[RE,LO]{Лекция АиСД 02.06}

\begin{document}

\paragraph{Ро-метод Полларда.}

$$\mathbb{Z}_n = \{0, 1, \ldots, n - 1\}$$
$$\mathbb{Z}_n^+ = \{1, 2, \ldots, n - 1\}$$
$$\mathbb{Z}_n^* = \{z \in \mathbb{Z}_n^+\,:\,gcd(z, n) = 1\}$$

Заметим, что для каждого числа количество чисел, не взаимно простых с составным $n$, хотя бы $O(\sqrt{n})$. Тогда можно было бы выбрать много случайных чисел и проверить, что $gcd(z, n) > 1$. Для каждого такого числа вероятность найти делитель будет $\frac{\sqrt{n}}{n} = \frac{1}{\sqrt{n}}$. Нам не хватает такой точности.

Построим функциональный граф для какой-то псевдослучайной функции $g$ (чаще всего $g(x) = x^2 + 1$ по модулю $n$) на остатках. Также навесим дополнительное требование на $g$, чтобы она сохранила остатки (то есть $g(x)\ mod\ a = g(x\ mod\ a)\ mod\ a$, наша функция этому удовлетворяет). Заметим, что в нем произвольный бесконечный путь будет выглядеть как буква $\rho$ --- сначала какой-то период, а потом цикл. С учетом случайности функции и парадокса дней рождения в этой букве $\rho$ будет $\sqrt{n}$ вершин.

Пока что мы еще ничего не выиграли, но осталось совсем немного. Возьмем и мысленно сделаем функциональные графы по всем остаткам меньше $n$ (назовем их $a$). При этом мы явно будем генерировать только путь в функциональном графе для числа $n$. Поскольку у составного $n$ был делитель меньше, чем $2\sqrt{n}$, то в каком-то функциональном графе мы зациклимся за $O(\sqrt[4]{n})$ шагов. Если мы возьмем все пары $(x_i, x_{2i})$, то тогда они за линейное время относительно размера буквы $\rho$ будут указывать на одинаковую вершину. А это будет значить, что $|x_i - x_{2i}| \equiv 0 \pmod{a}$. Тогда если $a$  было делителем $n$, то $gcd(|x_i - x_{2i}|, n) > 1$. Тогда мы нашли какой-то делитель и можно раскладывать рекурсивно.

\paragraph{Алгоритм Миллера-Рабина.} Создадим тест-проверку на $a^{n-1} \equiv 1 \pmod{n}$. Если это было верно для всех $a$ от 1 до $n - 1$, то число $n$ было простым, потому что тогда оно было взаимно простым со всеми $a < n$. Мы хотим брать случайные числа $a$, при этом сделать немного итераций. Это называется тестом Ферма.

Просто брать случайные $a$ нельзя --- есть числа Кармайкла, которые работают в качестве контртеста. Их какое-то полиномиальное количество, хоть и не очень много.

Сделаем новый тест: $a^2 \equiv 1 \pmod{n}, a \neq 1, a \neq -1$, таких чисел не бывает для простых $n$ (потому что это будет означать что $(a - 1)(a + 1) \equiv 0 \pmod{n}$)

Рассмотрим $n - 1 = 2^s \cdot k, k \equiv 1 \pmod{2}$.

Сделаем $A(x) = \{x^k, x^{2k}, x^{4k}, \ldots, x^{n - 1}\}$. Тогда если у нас есть пара соседей  $(d, 1), d \neq 1, d \neq -1$, то тогда наш второй тест провалился --- $n$ не простое. Если последним элементом последовательности было число $d \neq 1$, то провалился тест Ферма --- число составное. Назовем свидетелями такие $x$, для которых один из этих тестов выполнился, остальных назовем лжецами. Мы хотим показать, что при случайном выборе $x$-ов, вероятность получить лжеца будет не выше $\frac{1}{2}$.

Тогда $x \in \mathbb{Z}_n^+$, $\mathbb{Z}_n^+ = W \cup L$.
А еще запомним, что $\mathbb{Z}_n^*$ --- группа.

Пусть наше число не было числом Кармайкла. Тогда $\exists x \in \mathbb{Z}_n^*\,:\,x^{n-1} \not\equiv 1 \pmod{n}$. Тогда можно взять подгруппу $B = \{z \in \mathbb{Z}_n^* : z^{n-1} \equiv 1\}$ (она содержит единицу, )



План: Хотим показать, что $L \subseteq B \subseteq \mathbb{Z}_n^*$

Автор сломался понять доказательство, возможно затехает его позже.

\end{document}
