\documentclass{article}

\usepackage[top=0.6in, bottom=0.75in, left=0.625in, right=0.625in]{geometry}
\usepackage{amsmath,amsthm,amssymb,hyperref}
\usepackage[utf8x]{inputenc}
\usepackage[russianb]{babel}
\usepackage{hyperref}
% \usepackage{minted}
\usepackage{color}
\usepackage{fancyhdr}


\newcommand{\R}{\mathbb{R}}  
\newcommand{\Z}{\mathbb{Z}}
\newcommand{\N}{\mathbb{N}}
\newcommand{\Q}{\mathbb{Q}}
\newcommand{\tu}[1]{\underline{#1}}
\newcommand{\tb}[1]{\textbf{#1}}
\newcommand{\ti}[1]{\textit{#1}}
\newcommand{\aliq}{\mathrel{\raisebox{-0.5ex}{\vdots}}}
\newcommand{\mylim}[2]{\lim_{#1 \to #2}}
\newcommand{\abs}[1]{\left|{#1}\right|}
\newcommand{\brackets}[1]{\left({#1}\right)}
\newcommand{\sqbrackets}[1]{\left[{#1}\right]}

\newenvironment{theorem}[2][Theorem]{\begin{trivlist}
\item[\hskip \labelsep {\bfseries #1}\hskip \labelsep {\bfseries #2.}]}{\end{trivlist}}
\newenvironment{lemma}[2][Lemma]{\begin{trivlist}
\item[\hskip \labelsep {\bfseries #1}\hskip \labelsep {\bfseries #2.}]}{\end{trivlist}}
\newenvironment{claim}[2][Claim]{\begin{trivlist}
\item[\hskip \labelsep {\bfseries #1}\hskip \labelsep {\bfseries #2.}]}{\end{trivlist}}
\newenvironment{problem}[2][Problem]{\begin{trivlist}
\item[\hskip \labelsep {\bfseries #1}\hskip \labelsep {\bfseries #2.}]}{\end{trivlist}}
\newenvironment{proposition}[2][Proposition]{\begin{trivlist}
\item[\hskip \labelsep {\bfseries #1}\hskip \labelsep {\bfseries #2.}]}{\end{trivlist}}
\newenvironment{corollary}[2][Corollary]{\begin{trivlist}
\item[\hskip \labelsep {\bfseries #1}\hskip \labelsep {\bfseries #2.}]}{\end{trivlist}}

\newenvironment{solution}{\begin{proof}[Solution]}{\end{proof}}

\makeatletter
\newenvironment{sqcases}{%
  \matrix@check\sqcases\env@sqcases
}{%
  \endarray\right.%
}
\def\env@sqcases{%
  \let\@ifnextchar\new@ifnextchar
  \left\lbrack
  \def\arraystretch{1.2}%
  \array{@{}l@{\quad}l@{}}%
}
\makeatother

\ifx\pdfoutput\undefined
\usepackage{graphicx}
\else
\usepackage[pdftex]{graphicx}
\fi

\pagestyle{fancy}
\fancyhf{}
\fancyhead[LE,RO]{Амеличев Константин, ПМИ 191, @kik0s, \href{http://github.com/kik0s}{\textcolor{blue}{github}}, \href{http://codeforces.com/profile/kikos}{\textcolor{blue}{codeforces}}, \href{http://vk.com/i_tried_to_name_myself_kikos}{\textcolor{blue}{vk}}}
\fancyhead[RE,LO]{Лекция по АиСД 05.11}

\begin{document}


\textbf{Модели}

RAM-модель (Random Access Machine)
Вопросы, возникающие при создании модели
\begin{enumerate}
\item адресация 
\item какие инструкции
\item рекурсия
\item где лежат инструкции
\item размер данных 
\item кол-во памяти
\item случайность
\end{enumerate}

\paragraph{Адресация}

Есть ячейки, в которых можно хранить целые числа (ограничения на $MAXC$ разумные, и на них введена неявная адресация

\textit{Замечание.} Явная адресация --- при создании элемента получаем адрес и можем пользоваться только этим адресом. Неявно --- можем получать адреса каким-то своим образом, к примеру, $ptr + 20$.

\paragraph{Кол-во памяти}

Неявное соглашение RAM --- время работы не меньше памяти. По дефолту считаем, что мы его инициализируем мусором

\paragraph{Где инструкции}

Хранить инструкции можно в памяти и где-то снаружи. Мы будем хранить снаружи (внутри --- RASP-модель). Иначе говоря, инструкции и данные отделены. 


\paragraph{Какие инструкции}
В нашей модели есть инструкции следующих типов: 

\begin{itemize}
\item работа с памятью
\item ветвление
\item передача управления (=$goto$), 
\item арифметика (at least $a + b, a - b, \frac{a}{b}, \cdot, mod, \lfloor \frac{a}{b} \rfloor$)
\item сревнения (at least $a < b, a > b, a \le b, a \ge b, a = b, a \neq b$)
\item логические (at least $\land, \lor, \oplus, \lnot$)
\item битовые операции ($>>, <<, \&, |, \sim, \oplus$)
\item математические функции (опять-таки, в рамках разумного)
\item rand
\end{itemize}

Все инструкции работают от конечного разумного числа операндов (не умеем в векторные операции)

\paragraph{Размер данных}

$\exists~C, k~:~ C \cdot A^k \cdot n^k$ --- верхнее ограничение на величины промежуточных вычислений.

\paragraph{Рекурсия}

Рекурсия всегда линейна по памяити относительно глубины.

\paragraph{Случайность}

Мы считаем, что у нас есть абсолютно рандомная функция. Будем полагать, что у нас есть источник энтропии, выдающий случайности в промежутке $[0, 1]$.

\textbf{Время работы.} \begin{itemize}
\item наихудшее --- $t = \displaystyle\max_{input, random} t(input, random)$
\item наилучшее --- $t = \displaystyle\max_{input} \displaystyle\min_{random} t(input, random)$
\item ожидаемое --- $E\ t = \displaystyle\max_{input} {Average}_{random} t(input, random)$
\item на случайных данных --- $t = Average_{input} Average_{random} t(input, random)$
\end{itemize}


\textbf{Алгоритмы}

Методы доказательства корректности алгоритма.
\begin{enumerate}
    \item индукция
    \item инвариант
    \item от противного
\end{enumerate}

\end{document}