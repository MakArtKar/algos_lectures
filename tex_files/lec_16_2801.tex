\documentclass[12pt]{article}
\usepackage[utf8]{inputenc}
\usepackage[russian]{babel}
\usepackage[margin=1.5in,left=1cm,right=1cm, top=2cm,bottom=2cm,bindingoffset=0cm]{geometry}
\usepackage{graphicx}
\usepackage{color}
\usepackage{amssymb}
\usepackage{minted}
\usepackage{hyperref}
\usepackage{amsmath}
\usepackage{fancyhdr}



\title{Title}
\author{Амеличев Константин, ПМИ 191.}
\date{Date}

\newcommand{\problem}[2]{

\section {Задача #1}
\textbf {Постановка задачи.} {#2}

\textbf {Решение.}
}

\newcommand{\limit}[2]{\displaystyle \lim_{#1 \to #2}}

\newcommand{\rangesum}[2]{\displaystyle \sum_{#1}^{#2}}

\newcommand{\mintedparams}{
% frame=lines
% framesep=2mm,
% baselinestretch=1.2,
% bgcolor=LightGray    
}

\pagestyle{fancy}
\fancyhf{}
\fancyhead[LE,RO]{Амеличев Константин, ПМИ 191, @kik0s, \href{http://github.com/kik0s}{\textcolor{blue}{github}}, \href{http://codeforces.com/profile/kikos}{\textcolor{blue}{codeforces}}, \href{http://vk.com/i_tried_to_name_myself_kikos}{\textcolor{blue}{vk}}}
\fancyhead[RE,LO]{Лекция АиСД 28.01}

\begin{document}

\paragraph{Задачи оптимизации.} Во всех предыдущих задачах на структуры данных мы работали, оптимизируя какую-либо очевидную задачу (например, сумму на отрезке). Теперь же мы будем решать задачи, которые непонятно как решать, кроме как полным перебором всех вариантов ответа.

В качестве примера возьмем задачу $subset\ sum$. В ней нам нужно найти подмножество заданного множества с фиксированной суммой $S$. Если перебрать все подмножества, и посчитать сумму по каждому подмножеству. 

Перебором можно решить любую задачу, если перебрать все возможные ответы (множество вещественных чисел для нас тоже конечно!).

Подмножества хочется как-то пронумеровать. К сожалению, непонятно как закодировать $2^n$ чисел, если раньше мы разрешали в $RAM$-модели числа до $C^k \cdot n^k \cdot A$. Поэтому мы просто уточним $RAM$-модель, и разрешим $C^k \cdot t(n)^k \cdot A$ (Считая $t(n)$ таким временем работы, что внутри нет длинной арифметики).

\paragraph{Динамическое программирование.} Иногда бывает полезно запоминать промежуточные величины перебора. Более того, часто перебор можно <<ужать>>, если нам в переборе нужны не все величины (например, в задаче <<subset sum>> достаточно помнить только общую сумму, если перебирать элементы по очереди).

Сделаем $dp(i, x) \in \{0, 1\}$, которая будет говорить, можно ли набрать сумму $x$ с помощью первых $i$ элементов. Тогда $dp(i, x)$ можно пересчитать через $dp(i - 1, x)$ и $dp(i - 1, x - w_i)$.

Требования к нашей динамике:
\begin{itemize}
    \item Граф вычислений ацикличен.
    \item <<Состояния>> динамики явно задают нам всю необходимую информацию.
\end{itemize}

\paragraph{Задача о рюкзаке.} Пусть нам заданы $n,\ S,\ w_i,\ c_i$ (то есть элементы с весами и стоимостями). Мы хотим выбрать некоторое подмножество с суммарным весом не более $S$ и максимальной суммой стоимостей.

Мы можем сделать динамику $dp(i, w, c) \in \{0, 1\}$, которая решит нашу задачу. Но заметим, что наше решение монотонно по параметру $c$ (то есть, для равных $i$ и $w$ стоит отдавать предпочтение ответу с максимальным $c$). Тогда $c$ можно сделать \textit{значением} динамики. То есть, пересчитывать динамику $dp(i,w) \in C$ как максимум из $dp(i - 1, w)$ и $dp(i - 1, w - w_i) + c_i$. Кстати, заметим, что тут задача монотонна по всем параметрам сразу.

\paragraph{Задача коммивояжера (TSP).} Заданы точки на плоскости. Надо найти кратчайший кольцевой маршрут, проходящий по всем точкам хотя бы единожды.

Есть очевидное решение за $O((n-1)!)$. Воспользуемся ДП по подмножествам, основная идея которого --- понять, что нам в состоянии важнее всего только то, в каком \textit{множестве} вершин мы уже были, и в каких вершинах мы уже оказались. Закодировать множество мы можем с помощью двоичной маски. Решение с такой идее отработает уже за $O(2^nn^2)$.

\paragraph{ДП по подстрокам.} Отдельный трюк, когда подстроки пересчитываются через свои подотрезки. Важное отличие в том, что мы можем пересчитываться через несколько задач сразу ($dp(l, r) = dp(l, k) + dp(k, r)$), а за счет этого порядок пересчета на графе может быть неочевидным.

\end{document}
