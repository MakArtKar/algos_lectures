\documentclass{article}
\usepackage[T2A]{fontenc}
\usepackage[utf8]{inputenc}
\usepackage[russian]{babel}
\usepackage[pdftex]{graphicx}
\usepackage{amsfonts}
\usepackage{amsmath}
\usepackage{fancyhdr}


\author{Иваник Даниил}
\title{Лекция12}
\pagestyle{fancy}
\fancyhf{}
\fancyhead[LE,RO]{Иваник Даниил}
\fancyhead[RE,LO]{Семинар АиСД 17.12}
\begin{document}
Научимся в детермиированный, так сказать, декартач.

2-3 дерево

В этом дереве есть вершины степени 2 и вершины степени 3. Элементы записываются только в листьях.

Ряд условий 2-3 дерева:

1)Все листья на одной и той же глубине.
2)Степень каждой внутренней вершины либо 2, либо 3.

В промежуточных вершинах храним ключи, которыми можно разделять вершины-сыновей.

У вершины с 3 детьми, два ключа - максимум поддеревьев двух первых детей, с 2 детьми - максимум в первом (а ещё храним максимум во всём поддереве).

Смотря на ключ, мы можем понять в какое из двух (или трёх) деревьев идти. 

Вставка в 2-3 дерево.

Найдём место, в котором элемент должен быть и просто вставим его. Может случиться, что детей уже было 3, а теперь стало 4. Тогда разобьём нашу вершину (у которой стало 4 сына) и разобьём её на 2 и переподвесим их теперь к её матушке. Ну и продолжаем это процесс пока не дойдём до корня. Если вдруг мы рассплитили корень, то создадим новый корень и подвесим к нему наши две вершины. После того, как мы всё сделали, можно запуститься рекурсивно вверх от только что вставленной и идём вверх. Это работает быстро, потому что уровней логарифм.

Удаление из 2-3 дерева.
Найдём элемент и тупо удалим его. Если у его предка было три ребёнка - всё хорошо. Предположим, что у вершины было 2 ребёнка, а остался один (трагично, не правда ли?). Теперь начинаются боль и мучения. В общем тут перебор случаев, который можно разобрать кроме того случая, когда у нас только один брат, у которого ровно два сына.  Тогда мы переподвешиваем нашего сына к брату и переходим на уровень выше. Если мы упрёмся в корень, то просто возьмём корень и удалим его за ненадобностью.

Всё 2-3 дерево можно провязать двусвязными списками, связывающие вершины на одном уровне.

Почуму бы вместо 2-3 дерева не сделать бы 5-11 дерево? Оценим это так. Пусть у всех вершин степень d. Тогда стоимость операции:

\[d \cdot \log_d{n} = d \cdot \frac{\ln{n}}{\ln{d}} \Rightarrow\]
\[(d \cdot \log_d{n})^{'} = (d \cdot \frac{\ln{n}}{\ln{d}})^{'} \Rightarrow\]
\[(d \cdot \log_d{n})^{'} = (d \cdot \frac{\ln{d} - 1}{(\ln{d})^2})^{'}\]

0 производной в \(d = e\). Значит, 2 и 3 - то хорошие приближения.

\(B+\) - дерево.

Степень каждой вершины от \(t\) до \(2t - 1\).
Степень корня - от \(2\) до \(2t - 1\).

Поиск вершины - также.

Такие деревья нужны для алгоритмов во внешней памяти. 

Тогда ассимптотика - \(O(CPU \cdot (d \log_d{n}) + HDD \cdot \log_d{n})\).
Тогда, хочется брать достаточно большой \(d\). 

Вставка:

Ищем место для вставки и смотрим, куда вставить. Если в какой-то момент стало \(2t\) детей, то сплитим на \(t\) и \(t\) и переподвешиваем.

Удаление:
Пусть в вершине теперь стало \(t - 1\) детей (иначе всё хорошо). Если у нас соседний брат - "Большой Брат" (хотя бы \(t + 1\) ребёнок), то всё - мы нашли жертву, которую будем грабить (переподвешиваем одного из сыновей брата к нашей вершине). Если нет, то вдохновившись небезызвестным произведением Кена Кизи, подкидываем брату \(t - 1\) кукушонка. Переходим на уровень выше и продолжаем процесс рекурсивно. Отдельно оговорим корень. Тут проблема только тогда, когда у него осталась 1 вершина, но тогда корень улетает в небытие, и в этом городе появляется новый корень (время для нового Жожо). 

А сейчас будет нерекурсивная вставка в \(B\)-дерево.
Ослабим ограничения на вершины теперь границы это \(t - 1\) и \(2t - 1\). Во время поиска вершины будут разбиваться в момент спуска. Как только вершина может переполниться (её степень \(2t - 1\)), тогда разобьём нашу вершину на \(t - 1\) и \(t\) и пойдём дальше вниз.

Ещё одна отсечка, от которой плкавится мозг - в каждой вершине нам надо хранить массив максимумов детей. Будем хранить эту вещь бинарным деревом поиска (ДД).

\end{document}