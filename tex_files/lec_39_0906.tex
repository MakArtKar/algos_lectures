\documentclass[12pt]{article}
\usepackage[utf8]{inputenc}
\usepackage[russian]{babel}
\usepackage[margin=1.5in,left=1cm,right=1cm, top=2cm,bottom=2cm,bindingoffset=0cm]{geometry}
\usepackage{graphicx}
\usepackage{color}
\usepackage{amssymb}
\usepackage{minted}
\usepackage{hyperref}
\usepackage{amsmath}
\usepackage{fancyhdr}



\title{Title}
\author{Амеличев Константин, ПМИ 191.}
\date{Date}

\newcommand{\problem}[2]{

\section*{Задача #1}
\textbf {Постановка задачи.} {#2}

\textbf {Решение.}
}

\newcommand{\limit}[2]{\displaystyle \lim_{#1 \to #2}}

\newcommand{\rangesum}[2]{\displaystyle \sum_{#1}^{#2}}

\newcommand{\rangeint}[2]{\displaystyle \int_{#1}^{#2}}


\newcommand{\mintedparams}{
% frame=lines
% framesep=2mm,
% baselinestretch=1.2,
% bgcolor=LightGray    
}

\pagestyle{fancy}
\fancyhf{}
\fancyhead[LE,RO]{Амеличев Константин, ПМИ 191, @kik0s, \href{http://github.com/kik0s}{\textcolor{blue}{github}}, \href{http://codeforces.com/profile/kikos}{\textcolor{blue}{codeforces}}, \href{http://vk.com/i_tried_to_name_myself_kikos}{\textcolor{blue}{vk}}}
\fancyhead[RE,LO]{Лекция АиСД 09.06}

\begin{document}

\paragraph{Задача шифрования.} В общем задача выглядит так: есть Алиса и Боб, которые хотят передавать друг другу информацию таким образом, чтобы их сообщение можно было перехватить, но перехвативший человек не понял бы, что в нем написано. Дальше будем считать, что канал между Алисой и Бобом прослушивается Евой. В принципе, Евой может быть кто угодно, потому что каналы публичные. Причем, даже если Ева прочитает и расшифрует сообщение, Алиса с Бобом даже не узнают про это.

\paragraph{Схема RSA.} Пусть у Алисы и Боба было по два объекта: публичный и приватный ключ. Ключ --- это просто какой-то шифровальный объект. При этом ключи взаимообратимы, но для каждого из них по отдельности найти обратную функцию сложно, то есть $M = private(public(M)) = public(private(M))$. Если все участники знают публичные ключи, то обмен можно провести таким образом: Алиса произносит $public_B(M)$, тогда Боб делает $private_B(public_B(M)) = M$. 

Факты для RSA:
\begin{itemize}
\item $\forall a,n:\ a^{\phi(n)} \equiv 1 \pmod{n}$
\item $(a, b) = 1 \rightarrow \phi(ab) = \phi(a)\phi(b)$
\item Можно проверить число на простоту за $O(\log^k n)$
\item Нельзя посчитать факторизацию $n$ за $O(\log^k n)$. Этот факт не доказан человечеством.
\end{itemize}


Зафиксируем два простых ключа $p_1, p_2$. Их произведение равно $n$. Мы знаем, чему равно $\phi(n)$ --- это $(p_1 - 1)(p_2 - 1)$. При этому быстро посчитать $\phi(n)$ нельзя по факту 4.

Нашим публичным ключом будет пара $(e, n)$, приватным ключом будет пара $(d, n)$. $e$ и $d$ --- это такие числа, что $ed \equiv 1 \pmod{\phi(n)}$. Тогда $M^{ed} = M$. Тогда Алиса сначала фиксирует какое-то $e$, взаимно простое с $\phi(n)$. Тогда Алиса может решить диофантово уравнение $ed + \phi(n)m = 1$, получая $d$.

Теперь пусть все знают $e$. Тогда Алиса посылает $M^e$, Боб делает $M^ed = M^{\phi(n)k + 1} = M$.

\paragraph{Man in the middle и authority.} На самом деле, описанный выше не работает, если у нам есть Мэллори, которая читает и модифицирует канал связи между Алисой и Бобом. Мэллори может выдать им свои публичные ключи в качестве публичного ключа друг друга. Тогда Мэллори будет явно читать весь канал между Алисой и Бобом, причем поскольку она знает публичные ключи Алисы и Боба, она может отправлять им что угодно.

Таким образом, мы хотим решить следующую задачу: получить публичный ключ, которому можно доверять.

Мы предположим, что у нас есть добрый Трент, которому все стороны доверяют, и публичный ключ которого известен всем. Тогда все могут сообщить Тренту свои публичные ключи и спросить у Трента чужие публичные ключи.

\paragraph{Цифровая подпись.} Для того, чтобы все точно верили, что сообщение отправляет именно Трент, он может отправить пару $(M, private(M))$. Это называется цифровой подписью Трента. Теперь наши участники запрашивают сертификат у Трента, а Трент в качестве сертификата выдает Алисе свою цифровую подпись для этого ключа. То есть теперь Алиса может отправлять свой публичный ключ, представляясь Алисой, которую одобрил Трент.


\end{document}
