\documentclass[12pt]{article}
\usepackage[utf8]{inputenc}
\usepackage[russian]{babel}
\usepackage[margin=1.5in,left=1cm,right=1cm, top=2cm,bottom=2cm,bindingoffset=0cm]{geometry}
\usepackage{graphicx}
\usepackage{color}
\usepackage{amssymb}
\usepackage{minted}
\usepackage{hyperref}
\usepackage{amsmath}
\usepackage{fancyhdr}



\title{Title}
\author{Амеличев Константин, ПМИ 191.}
\date{Date}

\newcommand{\problem}[2]{

\section*{Задача #1}
\textbf {Постановка задачи.} {#2}

\textbf {Решение.}
}

\newcommand{\limit}[2]{\displaystyle \lim_{#1 \to #2}}

\newcommand{\rangesum}[2]{\displaystyle \sum_{#1}^{#2}}

\newcommand{\rangeint}[2]{\displaystyle \int_{#1}^{#2}}


\newcommand{\mintedparams}{
% frame=lines
% framesep=2mm,
% baselinestretch=1.2,
% bgcolor=LightGray    
}

\pagestyle{fancy}
\fancyhf{}
\fancyhead[LE,RO]{Амеличев Константин, ПМИ 191, @kik0s, \href{http://github.com/kik0s}{\textcolor{blue}{github}}, \href{http://codeforces.com/profile/kikos}{\textcolor{blue}{codeforces}}, \href{http://vk.com/i_tried_to_name_myself_kikos}{\textcolor{blue}{vk}}}
\fancyhead[RE,LO]{Лекция АиСД 16.04}

\begin{document}

\paragraph{Алгоритм Диница с масштабированием.} Аналогично Форду-Фалкерсону с масштабированием, будем по очереди запускать алгоритм Диница, ставя ограничения на величину остаточных пропускных способностей, необходимых для того, чтобы ребро попадало в остаточную сеть. Общий вид алгоритма будет выглядеть как-то так:

\begin{minted}{c++}
int max_flow() {
    for (SCALE = 1 << 30; SCALE > 0; SCALE >>= 1) {
        while (bfs(s, t)) {
            while(path = dfs(s, t)) {
                relax(path);
            }
        }
    }
}
\end{minted}

Какую мы получим выгоду от масштабирования? При переходе между итерациями мы знаем, что величина текущего минимального разреза не больше, чем $2 \cdot SCALE \cdot E$. То есть мы знаем, что теперь у нас будет суммарно не более $O(E)$ насыщений в рамках одной итерации.

Раньше мы оценивали Диница так: у нас был $dfs$, в котором мы считали число успешных и неуспешных итераций $a_i$ и $b_i$. Мы получали, что $\sum a_i = O(VE),\ \sum b_i = O(E)$. Сделаем новую оценку, используя тот факт, что насыщений всего $O(E)$. Здесь $t_d$ --- число путей длины $d$.

$$\rangesum{d=0}{V} \rangesum{i=0}{t_d} (a_{d, i} + b_{d, i}) \le \rangesum{d=0}{V} (d \cdot t_d + E) \le VE + V \cdot \rangesum{d=0}{V} t_d = O(VE)$$.

В итоге на каждой итерации масштабирования мы сделаем $O(VE)$ шагов, получаем оценку на время работы $O(VE \log C)$.

\paragraph{Оценки Карзанова.} Пусть в нашей сети единичные пропускные способности (например, в задаче о поиске максимального паросочетания). Тогда утверждается, что алгоритм Диница отработает за $O(E \sqrt{V})$.

Как это показать? Сначала покажем, что в рамках одной слоистой сети алгоритм делает $O(E)$ шагов. Для этого надо оценить сумму $a_i$, которая не превышает $E$, потому что после каждого успешного действия с ребром оно насыщается (читай --- удаляется).

Отдельно покажем, что если определить $P = \sum p(v, u)$ (где $p(v, u)$ --- потенциал вершины, сколько через нее максимально может пройти), то алгоритм Диница будет работать за $O(\sqrt{P} VE)$.

Определим вершинный разрез как такое множество вершин, через которое будет проходить любой путь из $s$ в $t$. 

Сделаем первые $\sqrt{P}$ итераций Диница. Теперь в нашей сети будет не меньше, чем $\sqrt{P}$ слоев. Поскольку эти слои не пересекались, то оставшийся поток в сети не больше минимума потока через вершинные разрезы, то есть он не больше, чем $\sqrt{P}$. А значит, алгоритм Диница найдет не более чем $\sqrt{P}$ путей. Таким образом, мы получим сложность $O(\sqrt{P} VE)$. В применении к задаче о паросочетании $P = V$, а в рамках одной слоистой сети происходит не $O(VE)$, а $O(E)$ шагов, откуда получается сложность $O(E \sqrt{V})$.

\paragraph{Стоимостные потоки.} Введем $w$ --- дополнительную стоимость потока вдоль ребра. Мы будем считать, что $w_{v, u} = -w_{u, v}$. Общей стоимостью назовем $\sum \frac{f_{v, u} \cdot w_{v, u}}{2}$ (у нас дважды учитывается поток вдоль одного ребра из-за обратных ребер). Обычно стоимость хотят минимизировать.

Выделяют несколько задач: min-cost-flow, min-cost-k-flow, min-cost-max-flow. В первой мы минимизируем вес, во второй мы ищем минимальный вес потока величины $k$, а в третьей найти максимальный поток, а среди таких потоков поток минимальной стоимости. Есть еще задача о циркуляции минимальной стоимости, к которой все эти задачи сводятся, если замкнуть $t \rightarrow s$ правильно заданным ребром.

\paragraph{Критерий оптимальности потока.} Поток считается минимальным по стоимости потоком размера $k$, если и только если в его остаточной сети нет циклов отрицательного веса. Почему? Ну пусть мы нашли какой-то другой поток размера $k$, который имел меньшую стоимость. Если рассмотреть $f_1 - f_2$, то это будет циркуляцией. Если рассмотреть ее декомпозицию, в ней будут только циклы, при этом стоимость отрицательная. Значит, какой-то из циклов имеет отрицательную стоимость. Противоречие.

В доказательстве было узкое место: у нас могли нарушиться условия на пропускные способности, и поэтому не факт, что мы можем добавить этот новый цикл. Но на самом деле это правда, потому что $f_1$ и $f_2$ были корректными потоками. А именно, при добавлении у нас поток по ребру будет не больше чем максимум из потоков по ребру, что остается корректным потоком.

\paragraph{Алгоритм поиска min-cost-k-flow.} Если у нас был поток $f$ минимальной стоимости веса $k$, то поток минимальной стоимости веса $k+1$ можно получить вдоль кратчайшего в плане стоимости пути в остаточной сети $G_f$. А это значит, что если мы сначала найдем минимальный поток веса 0 (брать отрицательные циклы, пока можем), а затем по очереди брать кратчайшие пути $p$ из $s$ в $t$.

Почему утверждение верно? Пусть был какой-то другой поток лучше. Тогда это значит? Что $w(f_1) + w(p) > w(f_2)$. Рассмотрим разность между этими потоками. В ней есть отрицательный цикл $c$. Рассмотрим $p + c$. Если его декомпозировать, то получим пути, каждый из которых не дешевле, чем $p$ (потому что $p$ был кратчайшим), и циклы, которые имели неотрицательную стоимость (иначе ими можно было бы уменьшить $f_1$). Тогда это значит, что поток $p + c$ имеет стоимость не меньше, чем $p$, противоречие.


\end{document}
