\documentclass[12pt]{article}
\usepackage[utf8]{inputenc}
\usepackage[russian]{babel}
\usepackage[margin=1.5in,left=1cm,right=1cm, top=2cm,bottom=2cm,bindingoffset=0cm]{geometry}
\usepackage{graphicx}
\usepackage{color}
\usepackage{amssymb}
\usepackage{minted}
\usepackage{hyperref}
\usepackage{amsmath}
\usepackage{fancyhdr}



\title{Title}
\author{Амеличев Константин, ПМИ 191.}
\date{Date}

\newcommand{\problem}[2]{

\section {Задача #1}
\textbf {Постановка задачи.} {#2}

\textbf {Решение.}
}

\newcommand{\limit}[2]{\displaystyle \lim_{#1 \to #2}}

\newcommand{\rangesum}[2]{\displaystyle \sum_{#1}^{#2}}

\newcommand{\mintedparams}{
% frame=lines
% framesep=2mm,
% baselinestretch=1.2,
% bgcolor=LightGray    
}

\pagestyle{fancy}
\fancyhf{}
\fancyhead[LE,RO]{Амеличев Константин, ПМИ 191, @kik0s, \href{http://github.com/kik0s}{\textcolor{blue}{github}}, \href{http://codeforces.com/profile/kikos}{\textcolor{blue}{codeforces}}, \href{http://vk.com/i_tried_to_name_myself_kikos}{\textcolor{blue}{vk}}}
\fancyhead[RE,LO]{Лекция АиСД 19.11}

\begin{document}

\section{Простые структуры данных}

\paragraph{Требования к структуре данных:}
\begin{itemize}
    \item От СД мы хотим обработку каких-то наших запросов.
    \item online-offline. Бывает, что мы знаем все запросы, бывает, что мы узнаем запрос только тогда, когда отвечаем на предыдущий
    \item При  обсуждении времени работы отделяется время на препроцессинг и на последующие ответы на запросы (query).
\end{itemize}

\subsection {Data structure / interface}
\hspace{\fill}
\textit{Структура данных} --- это какой-то математический объект, который умеет отвечать на наши запросы конкретным способом. Красно-черное дерево --- это структура данных.
\\
\textit{Интерфейс} --- это объект, с которым может взаимодействовать пользователь, который каким-то образом умеет отвечать на наши запросы (пользователю все равно, как, его волнует только то, что интерфейс реализует, и за какое время(память) он это делает). $std::set$ --- это интерфейс.
\\
\textit{Итератор} --- это специальный объект, отвечающий непосредственно за ячейку в структуре данных. Для того, чтобы удалить элемент, мы должны иметь итератор на этот элемент. Т.е. удаление по ключу работает за $O(erase)$, а удаление по значению за $O(find) + O(erase)$

\paragraph{list}
Списки бывают двусвязными, односвязными, циклическими. У каждого элемента есть ссылка на следующий (и иногда на предыдущий), а также есть отдельный глобальный указатель на начало списка.


\paragraph{stack}
Стек --- структура данных, которая умеет делать добавление в конец, удаление из конца, взятие последнего элемента, за $O(1)$.

\paragraph{queue}
Очередь --- структура данных, которая умеет делать добавление в конец, удаление из начала, взятие первого элемента, за $O(1)$.

\paragraph{deque}
Двусторонняя очередь --- структура данных, которая умеет делать добавление, удаление и взятие элемента с любого конца последовательности, за $O(1)$.

\paragraph{priority\_queue}
Очередь с приоритетами ака куча --- структура данных, которая умеет делать добавление, удаление, и быстрые операции с минимумом, представляющая из себя дерево с условием $parent(u) = v \rightarrow value(v) \le value(u)$.


Все эти структуры реализуются как на массиве (храним последовательную память и указатели на начало/конец), так и на списках (что на самом деле является тем же самым, что и на массиве, просто ссылка вперед эквивалентна $a_i \rightarrow a_{i + 1}$ в терминологии массивов, \textit{прим. автора}).

Структура данных на массиве кратно быстрее аналогичной на ссылках, потому что массив проходится по кэшу и не требует дополнительной памяти.

\begin{center}
\begin{tabular}{c c c c c c c c c}
data structure & add & delete & pop & find & top & build & min & get by index\\
stack & $O(1)$ & - & $O(1)$ & - & $O(1)$ & $O(n)$ & $O(n)$ & -\\
dynamic array & $O(1)$ & $O(n)$ & $O(1)$ & $O(n)$ & $O(1)$ & $O(n)$ & $O(n)$ & $O(1)$ \\
queue & $O(1)$ & - & $O(1)$ & - & $O(1)$ & $O(n)$ & $O(n)$ & -\\
deque & $O(1)$ & - & $O(1)$ & - & $O(1)$ & $O(n)$ & $O(n)$ & $O(1)$\\
linked list & $O(1)$ & $O(1)$ & $O(1)$ & $O(n)$ & $O(1)$ & $O(n)$ & $O(n)$ & $O(n)$\\
sorted array& $O(n)$ & $O(n)$ & $O(1)$ & $O(\log n)$ & $O(1)$ & $O(n \log n)$ & $O(1)$ & $O(1)$ \\
priority\_queue& $O(\log n)$ & $O(\log n)$ & $O(\log n)$ & $O(1)$ & $O(n)$ & $O(1)$ & -
\end{tabular}
\end{center}

\subsection{Двоичная куча}

Реализация двоичной кучи на массиве --- создаем массив размера $sz$, и создаем ребра $i \rightarrow 2 \cdot i,\ i \rightarrow 2 \cdot i + 1$.

От такой кучи мы хотим:
\begin{itemize}
\item insert(x)
\item get\_min()
\item extract\_min()
\item erase
\item change = \{decrease\_key, increase\_key\}
\end{itemize}

Для такой кучи мы реализуем $sift\_up(x),\ sift\_down(x)$ --- просеивание вниз и вверх. Процедура должна устранить конфликты с элементом $x$. Остальные операции умеют реализовываться через нее. 

$$insert = add\_leaf + sift_up$$

$$extract\_min = swap(root, last) + last -= 1 + sift_down(root)$$

$$erase = decrease\_key(-\infty) + extract\_min$$

Отдельно отметим построение кучи за $O(n)$ --- sift\_down поочередно для всех элементов $n,~n~-~1,~\dots,~1$.

\begin{minted}{c++}
void sift_up(int v) { // v >> 1 <=> v / 2
    if (key[v] < key[v >> 1]) {
        swap(key[v], key[v >> 1]);
        sift_up(v >> 1);
    }
}
\end{minted} 

\begin{minted}{c++}
void sift_down(int v, int size) { // indexes [1, size]
    if (2 * v > size) {
        return;
    }
    int left = 2 * v;
    int right = 2 * v + 1;
    int argmin = v;
    if (key[v] > key[left]) {
        argmin = left;
    }
    if (right <= size && key[right] < key[argmin]) {
        argmin = right;
    }
    if (argmin == v) {
        return;
    }
    swap(key[v], key[argmin]);
    sift_down(argmin, size);
}
\end{minted}

\end{document}
