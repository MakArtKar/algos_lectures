\documentclass{article}

\usepackage[top=0.6in, bottom=0.75in, left=0.625in, right=0.625in]{geometry}
\usepackage{amsmath,amsthm,amssymb,hyperref}
\usepackage[utf8x]{inputenc}
\usepackage[russianb]{babel}
\usepackage{hyperref}
% \usepackage{minted}
\usepackage{color}
\usepackage{fancyhdr}


\newcommand{\R}{\mathbb{R}}  
\newcommand{\Z}{\mathbb{Z}}
\newcommand{\N}{\mathbb{N}}
\newcommand{\Q}{\mathbb{Q}}
\newcommand{\tu}[1]{\underline{#1}}
\newcommand{\tb}[1]{\textbf{#1}}
\newcommand{\ti}[1]{\textit{#1}}
\newcommand{\aliq}{\mathrel{\raisebox{-0.5ex}{\vdots}}}
\newcommand{\mylim}[2]{\lim_{#1 \to #2}}
\newcommand{\abs}[1]{\left|{#1}\right|}
\newcommand{\brackets}[1]{\left({#1}\right)}
\newcommand{\sqbrackets}[1]{\left[{#1}\right]}

\newenvironment{theorem}[2][Theorem]{\begin{trivlist}
\item[\hskip \labelsep {\bfseries #1}\hskip \labelsep {\bfseries #2.}]}{\end{trivlist}}
\newenvironment{lemma}[2][Lemma]{\begin{trivlist}
\item[\hskip \labelsep {\bfseries #1}\hskip \labelsep {\bfseries #2.}]}{\end{trivlist}}
\newenvironment{claim}[2][Claim]{\begin{trivlist}
\item[\hskip \labelsep {\bfseries #1}\hskip \labelsep {\bfseries #2.}]}{\end{trivlist}}
\newenvironment{problem}[2][Problem]{\begin{trivlist}
\item[\hskip \labelsep {\bfseries #1}\hskip \labelsep {\bfseries #2.}]}{\end{trivlist}}
\newenvironment{proposition}[2][Proposition]{\begin{trivlist}
\item[\hskip \labelsep {\bfseries #1}\hskip \labelsep {\bfseries #2.}]}{\end{trivlist}}
\newenvironment{corollary}[2][Corollary]{\begin{trivlist}
\item[\hskip \labelsep {\bfseries #1}\hskip \labelsep {\bfseries #2.}]}{\end{trivlist}}

\newenvironment{solution}{\begin{proof}[Solution]}{\end{proof}}

\makeatletter
\newenvironment{sqcases}{%
  \matrix@check\sqcases\env@sqcases
}{%
  \endarray\right.%
}
\def\env@sqcases{%
  \let\@ifnextchar\new@ifnextchar
  \left\lbrack
  \def\arraystretch{1.2}%
  \array{@{}l@{\quad}l@{}}%
}
\makeatother

\ifx\pdfoutput\undefined
\usepackage{graphicx}
\else
\usepackage[pdftex]{graphicx}
\fi

\pagestyle{fancy}
\fancyhf{}
\fancyhead[LE,RO]{Макоян Артем, ПМИ 191-1, @MakArtKar, \href{http://github.com/MakArtKar}{\textcolor{blue}{github}}, \href{http://codeforces.com/profile/MakArtKar}{\textcolor{blue}{codeforces}}, \href{vk.com/makartkar}{\textcolor{blue}{vk}}}
\fancyhead[RE,LO]{Лекция по АиСД 22.10}

\begin{document}


\large

\tu{Формула оценки}: \tb{0.25 дз + 0.3 контест + 0.15 кр + 0.3 экзамен + Бонус}

\tu{Домашние задания}: сдавать устно(раз в неделю ассисты устраивают доп пару, запись онлайн) или latex, дедлайн 10-21 день.

\tu{Контесты}: Длинные(код ревью), короткие(раз в 2 недели), неточные, бонусные(идет к бонусу). Штрафов нет.

\tu{Контрольные работы}: раз в модуль, тестовые вопросы.

\tu{Бонусы}: бонусные контесты, ACM, работа на семинаре.

\tu{Материалы}:
\begin{itemize}
  \item Кормен
  \item en.wikipedia
  \item викиконспекты
  \item e-maxx
  \item Корте-Фанен Коммбинаторная оптимизация
\end{itemize}

\tb{Теория вероятности}.
$(\Omega, 2^{\Omega}, P)$ - вероятностная пространство.\\
$A \subset \Omega$, $P(A) = \sum_{w \in A} P(w)$. \\
\tu{Def}: $A, B$ - события, $P(B) > 0$. \tb{P(A|B)} - вероятность события A, если наступило событие B. Тогда $P(A|B) = \dfrac{\sum_{w \in A \cap B} P(w)}{\sum_{w \in B} P(w)} = \dfrac{P(A \cap B)}{P(B)}$. \\
\tu{Def}: A и B независимые, если P(A|B) = P(A). \\
Тогда, если A и B независимые, то $P(A \cap B) = P(A) \cdot P(B)$. \\



\end{document}