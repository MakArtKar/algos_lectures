\documentclass{article}

\usepackage[top=0.6in, bottom=0.75in, left=0.625in, right=0.625in]{geometry}
\usepackage{amsmath,amsthm,amssymb,hyperref}
\usepackage[utf8x]{inputenc}
\usepackage[russianb]{babel}
\usepackage{hyperref}
% \usepackage{minted}
\usepackage{color}
\usepackage{fancyhdr}


\newcommand{\R}{\mathbb{R}}  
\newcommand{\Z}{\mathbb{Z}}
\newcommand{\N}{\mathbb{N}}
\newcommand{\Q}{\mathbb{Q}}
\newcommand{\tu}[1]{\underline{#1}}
\newcommand{\tb}[1]{\textbf{#1}}
\newcommand{\ti}[1]{\textit{#1}}
\newcommand{\aliq}{\mathrel{\raisebox{-0.5ex}{\vdots}}}
\newcommand{\mylim}[2]{\lim_{#1 \to #2}}
\newcommand{\abs}[1]{\left|{#1}\right|}
\newcommand{\brackets}[1]{\left({#1}\right)}
\newcommand{\sqbrackets}[1]{\left[{#1}\right]}

\newenvironment{theorem}[2][Theorem]{\begin{trivlist}
\item[\hskip \labelsep {\bfseries #1}\hskip \labelsep {\bfseries #2.}]}{\end{trivlist}}
\newenvironment{lemma}[2][Lemma]{\begin{trivlist}
\item[\hskip \labelsep {\bfseries #1}\hskip \labelsep {\bfseries #2.}]}{\end{trivlist}}
\newenvironment{claim}[2][Claim]{\begin{trivlist}
\item[\hskip \labelsep {\bfseries #1}\hskip \labelsep {\bfseries #2.}]}{\end{trivlist}}
\newenvironment{problem}[2][Problem]{\begin{trivlist}
\item[\hskip \labelsep {\bfseries #1}\hskip \labelsep {\bfseries #2.}]}{\end{trivlist}}
\newenvironment{proposition}[2][Proposition]{\begin{trivlist}
\item[\hskip \labelsep {\bfseries #1}\hskip \labelsep {\bfseries #2.}]}{\end{trivlist}}
\newenvironment{corollary}[2][Corollary]{\begin{trivlist}
\item[\hskip \labelsep {\bfseries #1}\hskip \labelsep {\bfseries #2.}]}{\end{trivlist}}

\newenvironment{solution}{\begin{proof}[Solution]}{\end{proof}}

\makeatletter
\newenvironment{sqcases}{%
  \matrix@check\sqcases\env@sqcases
}{%
  \endarray\right.%
}
\def\env@sqcases{%
  \let\@ifnextchar\new@ifnextchar
  \left\lbrack
  \def\arraystretch{1.2}%
  \array{@{}l@{\quad}l@{}}%
}
\makeatother

\ifx\pdfoutput\undefined
\usepackage{graphicx}
\else
\usepackage[pdftex]{graphicx}
\fi

\pagestyle{fancy}
\fancyhf{}
\fancyhead[LE,RO]{Макоян Артем, ПМИ 191-1, @MakArtKar, \href{http://github.com/MakArtKar}{\textcolor{blue}{github}}, \href{http://codeforces.com/profile/MakArtKar}{\textcolor{blue}{codeforces}}, \href{vk.com/makartkar}{\textcolor{blue}{vk}}}
\fancyhead[RE,LO]{Семинар по АиСД 22.10}

\begin{document}


\large

t(n, input) - время работы алгоритмы при входных данных input размера n. Тогда время работы алгоритма $t(n) = max_{input} t(n, input)$. 

$t(n) = O(f(n)) \Leftrightarrow \exists c > 0, \; N > 0 : \forall n > N \; t(n) \leq c \cdot f(n)$.

$t(n) = o(f(n)) \Leftrightarrow \forall c > 0, \exists N : \forall n > N \; t(n) \leq c \cdot f(n)$.

$t(n) = \Omega(f(n)) \Leftrightarrow \exists c > 0, \; \forall N, \; \exists n > N, \; input : t(n, input) \geq c \cdot f(n) $.

$t(n) = \omega(f(n)) \Leftrightarrow \forall c > 0, \; \forall N, \; \exists n > N, \; input : t(n, input) \geq c \cdot f(n) $.

$t(n) = \theta(f(n)) \Leftrightarrow t(n) = \Omega(f(n)), \; t(n) = O(f(n))$.

Алгоритм является \tu{полиномиальным}, если $t(input) = O(\abs{input}^k)$. $\abs{input}$ - битовая длина.

\tu{Сильно полиномиальный алгоритм} - $t(n) = O(Poly(n))$ - string-poly

\tu{Слабо полиномиальный алгоритм} - $O(Poly(n, \log{C}))$ - weak-poly

\tu{Псевдо полиномиальный алгоритм} - $O(Poly(n, C))$ - pseudo-poly


\end{document}